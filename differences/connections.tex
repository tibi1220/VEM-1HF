% \begin{noindent}
\begin{luacode*}
  local connections = V.parametric.connections
  local len  = #connections

  for i=1,len do
    local type = connections[i].dof
    local loc = connections[i].l
    

    if i == 1 or i == 0 then
      tex.sprint "az "
    else 
      tex.sprint "a "
    end
    tex.sprint([[($]] .. loc:upper() .. [[$) pontban lévő]])

    if type == 1 then
      tex.sprint [[ görgős támasz]]
    elseif type == 2 then
      tex.sprint [[ csuklós támasz]]
    elseif type == 3 then
      tex.sprint [[ befogás]]
    end

    if i == len - 1 then
      tex.sprint [[, valamint  ]]
    elseif i == len then
      tex.sprint [[ miatt ]]
    else
      tex.sprint [[, ]]
    end
  end
\end{luacode*}
% \end{noindent}
az alábbi peremfeltételeket kapjuk:
\begin{equation}
  % \begin{noindent}
  \begin{luacode*}
    local notFree = V.parametric.notFree

    for k=1,3 do
      local i = notFree[k]
      local j = math.floor((i+1)/2)

      if i%2 == 1 then
        tex.sprint("v_" .. j .. "=0")
      else
        tex.sprint("\\varphi_" .. j .. "=0")
      end

      if k==3 then
        tex.sprint [[ \text. ]]
      else
        tex.sprint [[ \text,\quad ]]
      end
    end
  \end{luacode*}
  % \end{noindent}
\end{equation}
Ezen adatok ismeretében már fel tudjuk írni a kondenzált merevségi egyenletet,
melyet úgy kapunk meg, hogy a  gátolt szabadsági fokokhoz tartozó sorokat és
oszlopokat töröljük az eredeti, globális merevségi egyenletből:
\begin{equation}
  \sifigures{3} \sisci{}
  \underbrace{\begin{bmatrix}
      \directlua{H.printKkond()}
    \end{bmatrix}}_{\widehat{\rmat K}}
  % \begin{noindent}
  \begin{luacode*}
    tex.sprint [[\underbrace{\begin{bmatrix}]]
    local free = V.parametric.free

    for k=1,5 do
      local i = free[k]
      local j = math.floor((i+1)/2)

      if i%2 == 1 then
        tex.sprint("v_" .. j .. "\\\\")
      else
        tex.sprint("\\varphi_" .. j .. "\\\\")
      end
    end
    tex.sprint [[\end{bmatrix}}_{\widehat{\rvec U}}]]
  \end{luacode*}
  % \end{noindent}
  =
  \siplaces{2} \sifix{}
  \underbrace{\begin{bmatrix}
      \directlua{H.printFkond()}
    \end{bmatrix}}_{\widehat{\rvec F}}
  \text.
  \label{eq:cond}
\end{equation}
