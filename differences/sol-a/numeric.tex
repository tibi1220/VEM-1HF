% Az analitikus módszerrel kapott függvények értékei a nevezetes pontokban:
% \begin{noindent}
\begin{table}[H]
\centering
\caption{Az analitikus módszerrel kapott függvényértékek a nevezetes pontokban}
\def\arraystretch{1.25}
\begin{luacode*}
  tex.sprint[[\begin{tabular}{c|c|c|c|c|}]]
  tex.sprint[[& $x = 0$ & $x = a$ & $x = b$ & $x = c$ \\\hline]]

  tex.sprint [[$w(x)$]]

  for i=1,4 do
    local num = V.aNum[1][i]
    tex.sprint [[&]]
    if math.abs(num) < 1e-10 then
      tex.sprint [[$0\,\mathrm{m}$]]
    else
      H.printSIDirect { value = num, unit="m", dec=6 }
    end
  end

  tex.sprint [[\\]]

  tex.sprint [[$\vartheta(x)$]]

  for i=1,4 do
    local num = V.aNum[2][i]
    tex.sprint [[&]]
    if math.abs(num) < 1e-10 then
      tex.sprint [[$0\,\mathrm{rad}$]]
    else
      H.printSIDirect { value = num, unit="rad", dec=6 }
    end
  end

  tex.sprint [[\\]]

  tex.sprint [[$\spec M_h(x^-)$ & $-$]]

  for i=2,4 do
    local num = V.aNum[3][i]
    tex.sprint [[&]]
    if math.abs(num) < 1e-10 then
      tex.sprint [[$0\,\mathrm{Nm}$]]
    else
      H.printSIDirect { value = num, unit="Nm", dec=3 }
    end
  end

  tex.sprint [[\\]]

  tex.sprint [[$\spec M_h(x^+)$]]

  for i=1,3 do
    local num = V.aNum[4][i]
    tex.sprint [[&]]
    if math.abs(num) < 1e-10 then
      tex.sprint [[$0\,\mathrm{Nm}$]]
    else
      H.printSIDirect { value = num, unit="Nm", dec=3 }
    end
  end

  tex.sprint [[& $-$ \\\hline]]

  tex.sprint[[\end{tabular}]]
\end{luacode*}
\end{table}
% \end{noindent}
