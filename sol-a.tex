\subsubsection{Az egyensúlyi egyenletek felírása}

Írjuk fel az $y$ és $z$ irányú egyensúlyi egyenleteket:
\begin{align}
  \sum F_y = 0 \quad & \rightarrow \quad
  \F[+]{"O"}
  \F[+]{"A"}
  \F[+]{"B"}
  \F[+]{"C"}
  \F[-]{1}
  \F[-]{2}
  \F[-]{3}
  \p[-]{1}[l_1]
  \p[-]{2}[l_2]
  \p[-]{3}[l_3]
  = 0
  \\
  \sum M_O = 0 \quad & \rightarrow \quad
  \F[+]{"A"}[a]
  \F[+]{"B"}[b]
  \F[+]{"C"}[c]
  \M[+]{2}
  \M[+]{2}
  \M[+]{"C"}
  \F[-]{2}[a]
  \F[-]{3}[b]
  \p[-]{1}[l_1 \left(\frac{l_1}{2}\right)]
  \p[-]{2}[l_2 \left(a + \frac{l_2}{2}\right)]
  \p[-]{3}[l_3 \left(b + l_2 + \frac{l_3}{2}\right)]
\end{align}

\subsubsection{A rugalmas szál differenciálegyenlete}

Az egyenletrendszer nem megoldható, mivel 3 ismeretlenünk van, de csak 2
egyenletünk. Hívjuk segítségül a rugalmas szál differenciálegyenletét:
\begin{equation}
  -I_i E_i \cdot w_i''(x) = M_{h,i}(x)
  \text.
  \label{eq:diffeq-base}
\end{equation}
Határozzuk meg először a hajlítónyomatéki igénybevételi függvényt. Ehhez
a szerkezetet szakaszokra kell bontanunk. Akkor szükséges új szakaszt
kezdenünk, ha egy pontban keresztmetszet-változás, vagy koncentrált erőpár
lép fel, illetve ha megoszló terhelés kezdődik, vagy végződik. A szakaszok
tehát:
\begin{multicols}{3}
  \begin{enumerate}
    \item $x \in (0; a)$
    \item $x \in (a; b)$,
    \item $x \in (b; c)$,
  \end{enumerate}
\end{multicols}
Az egyes szakaszok igénybevételi függvénye tehát:
\begin{align}
  M_{h1}(x) & =
  \F[+]{1}[x]
  \F[-]{"O"}[x]
  \p[+]{1}[\frac{x^2}{2}]
  \text,
  \\
  M_{h2}(x) & =
  \F[+]{1}[x]
  \F[-]{"O"}[x]
  \F[+]{2}[(x - a)]
  \F[-]{"A"}[(x - a)]
  \M[+]{2}
  \p[+]{1}[a \left( x - \frac{a}{2} \right)]
  \p[+]{2}[ \frac{\left( x - a \right)^2 }{2}]
  \text,
  \\
  M_{h3}(x) & =
  \M[-]{"C"}
  \F[-]{"C"}[(c - x)]
  \p[+]{3}[\frac{\left( c - x \right)^2}{2}]
  \text.
\end{align}
Ezen függvények ismeretében a differenciálegyenletek:
\newcommand{\op}[1]{&{\textstyle \left/ #1 \right.}}
\newcommand{\opint}{\op{\int \mathrm d x}}


Első szakasz: $x \in (0;a)$
\begin{align}
  - \I{1} \E{1} w_1''(x) & =
  M_{h1}(x)
  \\
  - \I{1} \E{1} w_1''(x) & =
  \F[+]{1}[x]
  \F[-]{"O"}[x]
  \p[+]{1}[\frac{x^2}{2}]
  \op{\cdot(-1)}
  \\
  \I{1} \E{1} w_1''(x)   & =
  \F[-]{1}[x]
  \F[+]{"O"}[x]
  \p[-]{1}[\frac{x^2}{2}]
  \opint{}
  \\
  \I{1} \E{1} w_1'(x)    & =
  \F[-]{1}[\frac{x^2}{2}]
  \F[+]{"O"}[\frac{x^2}{2}]
  \p[-]{1}[\frac{x^3}{6}]
  + C_{11}
  \opint{}
  \\
  \I{1} \E{1} w_1(x)     & =
  \F[-]{1}[\frac{x^3}{6}]
  \F[+]{"O"}[\frac{x^3}{6}]
  \p[-]{1}[\frac{x^4}{24}]
  + C_{11} x
  + C_{12}
\end{align}

Második szakasz: $x \in (a;b)$
\par\vspace{-5mm}\KOMAoptions{fontsize=10pt}
\begin{align}
  - \I{2} \E{2} w_2''(x) & =
  M_{h2}(x)
  \\
  - \I{2} \E{2} w_2''(x) & =
  \F[+]{1}[x]
  \F[-]{"O"}[x]
  \F[+]{2}[(x - a)]
  \F[-]{"A"}[(x - a)]
  \M[+]{2}
  \p[+]{1}[a \left( x - \frac{a}{2} \right)]
  \p[+]{2}[ \frac{\left( x - a \right)^2 }{2}]
  \op{\cdot(-1)}
  \\
  \I{2} \E{2} w_2''(x)   & =
  \F[-]{1}[x]
  \F[+]{"O"}[x]
  \F[+]{2}[(-x + a)]
  \F[+]{"A"}[(x - a)]
  \M[-]{2}
  \p[+]{1}[\left(-a x + \frac{a^2}{2} \right)]
  \p[+]{2}[\left(-\frac{x^2}{2} + x a - \frac{a^2}{2}\right)]
  \opint
  \\
  \I{2} \E{2} w_2(x)     & =
  \F[-]{1}[\frac{x^2}{2}]
  \F[+]{"O"}[\frac{x^2}{2}]
  \F[+]{2}[\left(-\frac{x^2}{2} + ax \right)]
  \F[+]{"A"}[\left(\frac{x^2}{2} - ax \right)]
  \M[-]{2}[x]
  \p[+]{1}[\left(-\frac{a x^2}{2} + \frac{a^2 x}{2} \right)]
  \p[+]{2}[\left(-\frac{x^3}{6} + \frac{x^2 a}{2} - \frac{x a^2}{2}\right)]
  + C_{21}
  \opint
  \\
  \I{2} \E{2} w_2'(x)    & =
  \F[-]{1}[\frac{x^3}{6}]
  \F[+]{"O"}[\frac{x^3}{6}]
  \F[+]{2}[\left(-\frac{x^3}{6} + \frac{ax^2}{2} \right)]
  \F[+]{"A"}[\left(\frac{x^3}{6} - \frac{ax^2}{2} \right)]
  \M[-]{2}[\frac{x^2}{2}]
  \p[+]{1}[\left(-\frac{a x^3}{6} + \frac{a^2 x^2}{4} \right)]
  \p[+]{2}[\left(-\frac{x^4}{24} + \frac{x^3 a}{6} - \frac{x^2 a^2}{4}\right)]
  + C_{21} x
  + C_{22}
\end{align}

\KOMAoptions{fontsize=12pt}
Harmadik szakasz: $x \in (b;c)$
\begin{align}
  - \I{3} \E{3} w_3''(x) & =
  M_{h3}(x)
  \\
  - \I{3} \E{3} w_3''(x) & =
  \M[-]{"C"}
  \F[-]{"C"}[(c - x)]
  \p[+]{3}[\frac{\left( c - x \right)^2}{2}]
  \op{\cdot(-1)}
  \\
  \I{3} \E{3} w_3''(x)   & =
  \M[+]{"C"}
  \F[+]{"C"}[(c - x)]
  \p[+]{3}[\left(-\frac{x^2}{2} + cx - \frac{c^2}{2}\right)]
  \opint{}
  \\
  \I{3} \E{3} w_3'(x)    & =
  \M[+]{"C"}[x]
  \F[+]{"C"}[\left( cx - \frac{x^2}{2} \right)]
  \p[+]{3}[\left( -\frac{x c^2}{2} + \frac{cx^2}{2} - \frac{x^3}{6}\right)]
  + C_{31}
  \opint{}
  \\
  \I{3} \E{3} w_3(x)     & =
  \M[+]{"C"}[\frac{x^2}{2}]
  \F[+]{"C"}[\left( \frac{cx^2}{2} - \frac{x^3}{6} \right)]
  \p[+]{3}[\left( -\frac{x^2 c^2}{4} + \frac{cx^3}{6} - \frac{x^4}{24}\right)]
  + C_{31} x
  + C_{32}
\end{align}

A szögelfordulás függvény tehát:
\begin{align}
  w'_1(x) & = \frac{1}{\I{1}\E{1}} \left[
    \F[-]{1}[\frac{x^2}{2}]
    \F[+]{"O"}[\frac{x^2}{2}]
    \p[-]{1}[\frac{x^3}{6}]
    + C_{11}
    \right]
  \text,
  \\[3mm]
  w'_2(x) & = \frac{1}{\I{2}\E{2}} \left[
    \F[-]{1}[\frac{x^2}{2}]
    \F[+]{"O"}[\frac{x^2}{2}]
    \F[+]{2}[\left(-\frac{x^2}{2} + ax \right)]
    \F[+]{"A"}[\left(\frac{x^2}{2} - ax \right)]
    \M[-]{2}[x]
    \p[+]{1}[\left(-\frac{a x^2}{2} + \frac{a^2 x}{2} \right)]
    \p[+]{2}[\left(-\frac{x^3}{6} + \frac{x^2 a}{2} - \frac{x a^2}{2}\right)]
    + C_{21}
    \right]
  \text,
  \\[3mm]
  w'_3(x) & = \frac{1}{\I{3}\E{3}} \left[
    \M[+]{"C"}[x]
    \F[+]{"C"}[\left( cx - \frac{x^2}{2} \right)]
    \p[+]{3}[\left( -\frac{x c^2}{2} + \frac{cx^2}{2} - \frac{x^3}{6}\right)]
    + C_{31}
    \right]
  \text.
\end{align}

A lehajlás függvények pedig:
\begin{align}
  w_1(x) & = \frac{1}{\I{1}\E{1}} \left[
    \F[-]{1}[\frac{x^3}{6}]
    \F[+]{"O"}[\frac{x^3}{6}]
    \p[-]{1}[\frac{x^4}{24}]
    + C_{11} x
    + C_{12}
    \right]
  \text,
  \\[3mm]
  w_2(x) & = \frac{1}{\I{2}\E{2}} \left[
    \scalebox{.85}{\(\displaystyle%
      \F[-]{1}[\frac{x^3}{6}]
      \F[+]{"O"}[\frac{x^3}{6}]
      \F[+]{2}[\left(-\frac{x^3}{6} + \frac{ax^2}{2} \right)]
      \F[+]{"A"}[\left(\frac{x^3}{6} - \frac{ax^2}{2} \right)]
      \M[-]{2}[\frac{x^2}{2}]
      \p[+]{1}[\left(-\frac{a x^3}{6} + \frac{a^2 x^2}{4} \right)]
      \p[+]{2}[\left(-\frac{x^4}{24} + \frac{x^3 a}{6} - \frac{x^2 a^2}{4}\right)]
      + C_{21} x
      + C_{22}
      \)}
    \right]
  \text,
  \\[3mm]
  w_3(x) & = \frac{1}{\I{3}\E{3}} \left[
    \M[+]{"C"}[\frac{x^2}{2}]
    \F[+]{"C"}[\left( \frac{cx^2}{2} - \frac{x^3}{6} \right)]
    \p[+]{3}[\left( -\frac{x^2 c^2}{4} + \frac{cx^3}{6} - \frac{x^4}{24}\right)]
    + C_{31}x
    + C_{32}
    \right]
  \text.
\end{align}

\subsubsection{Az ismeretlenek meghatározása peremfeltételek segítségével}

A geometriai kényszerek, valamint a $w(x)$ illetve $\varphi(x)$ függvények
folytonosságából adódóan 7 peremfeltételt tudunk felírni:
% \begin{noindent}
\begin{luacode*}
  tex.sprint [[ \begin{center} \def\arraystretch{1.2} \begin{tabular}{r l} ]]
  tex.sprint [[ $\left.\begin{array}{c} w_1(a) = w_2(a) \\ w_2(b) = w_3(b) \end{array}\right\}$ & $w(x)$ folytonos, ]]
  tex.sprint [[ \\ $\left.\begin{array}{c} w_1'(a) = w_2'(a) \\ w_2'(b) = w_3'(b) \end{array}\right\}$ & $w'(x)$ folytonos, ]]

  local connections = V.parametric.connections
  local len  = #connections

  for i=1,len do
    local type = connections[i].dof
    local loc = connections[i].l

    if type == 1 then
      tex.sprint [[ \\$\left.\begin{array}{c} w( ]]
      tex.sprint(loc ~= 'o' and loc or 0)
      tex.sprint [[ ) = 0 \end{array}\right\}$ & görgős támasz]]
    elseif type == 2 then
      tex.sprint [[ \\$\left.\begin{array}{c} w( ]]
      tex.sprint(loc ~= 'o' and loc or 0)
      tex.sprint [[ ) = 0 \end{array}\right\}$ & csuklós támasz]]
    elseif type == 3 then
      tex.sprint [[ \\$\left.\begin{array}{c} w(c) = 0 \\ w'(c) = 0 \end{array}\right\}$ & befogás]]
    end

    if i == len then
      tex.sprint "."
    else
      tex.sprint ","
    end
  end

  tex.sprint [[ \end{tabular} \end{center} ]]
\end{luacode*}
% \end{noindent}


Így kilenc egyenletünk és kilenc ismeretlenünk van. Ez mátrixosan,
SI egységekben:
\scriptsize
\begin{equation}
  \bgroup
  \sifigures{2}
  \sisci{}
  \begin{bmatrix}
    \directlua{H.printAkond()}
  \end{bmatrix}
  \egroup
  \begin{bmatrix}
    \directlua{H.printXParamkond()}
  \end{bmatrix}
  =
  \siplaces{3}
  \begin{bmatrix}
    \directlua{H.printBkond()}
  \end{bmatrix}
  \text.
  \label{eq:alma}
\end{equation}
\normalsize

Az egyenletrendszer mátrix invertálással megoldható:
\scriptsize
\begin{equation}
  \begin{bmatrix}
    \directlua{H.printXParamkond()}
  \end{bmatrix}
  =
  \bgroup
  \sifigures{2}
  \sisci{}
  \begin{bmatrix}
    \directlua{H.printAi()}
  \end{bmatrix}
  \egroup
  \siplaces{3}
  \begin{bmatrix}
    \directlua{H.printBkond()}
  \end{bmatrix}
  \label{eq:alma-inverted}
  \text.
\end{equation}
\normalsize

Az egyenletrendszer megoldása numerikusan:
\begin{equation}
  \begin{bmatrix}
    \directlua{H.printXParamkond()}
  \end{bmatrix}
  =
  \sifigures{6}
  \begin{bmatrix}
    \directlua{H.printXkond()}
  \end{bmatrix}
  \mathrm{SI}
  \text.
  \label{eq:alma-sol}
\end{equation}

\subsubsection{Az analitikus módszerrel kapott függvények numerikusan}

Helyettesítsük be az előbb kiszámolt konstansokat a keresett függvényekbe.

A lehajlás függvény:
\bgroup
\siplaces{4}
% \begin{noindent}
\begin{luacode*}
  tex.sprint [[\begin{alignat}{9}]]
  for i=1,3 do
    tex.sprint("w_" .. i .. "(x)&=\\,")

    local vMat = V.wMat

    for k=4,0,-1 do
      local value = vMat[i][k]

      if math.abs(value) >= 1e-16 then
        if value > 0 and k ~= 3 then
          tex.sprint "&+&"
        else
          tex.sprint "&-&"
        end
        H.printSIDirect {value=math.abs(value), unit="", dec=6}

        tex.sprint "&\\,"
        if k == 0 then
          tex.sprint ""
        elseif  k== 1 then 
          tex.sprint "x"
        else
          tex.sprint("x^" .. k)
        end
        tex.sprint("\\,")
      else
        tex.sprint("&&&")
      end
    end

    tex.sprint("&[\\mathrm m] \\text{,\\quad ha} \\quad x \\in (" .. ({0, 'a', 'b'})[i] .. ";" .. ({'a', 'b', 'c'})[i] .. ")")

    if i ~= 3 then
      tex.sprint "\\text,\\\\"
    else
      tex.sprint "\\text."
    end
  end
  tex.sprint [[\end{alignat}]]
\end{luacode*}
% \end{noindent}
\egroup

A szögelfordulás függvény:
\bgroup
\siplaces{6}
% \begin{noindent}
\begin{luacode*}
  tex.sprint [[\begin{alignat}{9}]]
  for i=1,3 do
    tex.sprint("\\vartheta_" .. i .. "(x)&=\\,")

    local phiMat = V.thetaMat

    for k=3,0,-1 do
      local value = phiMat[i][k]

      if math.abs(value) >= 1e-16 then
        if value > 0 and k ~= 3 then
          tex.sprint "&+&"
        else
          tex.sprint "&-&"
        end
        H.printSIDirect {value=math.abs(value), unit="", dec=6}

        tex.sprint "&\\,"
        if k == 0 then
          tex.sprint ""
        elseif  k == 1 then 
          tex.sprint "x"
        else
          tex.sprint("x^" .. k)
        end
        tex.sprint("\\,")
      else
        tex.sprint("&&&")
      end
    end

    tex.sprint("[&\\mathrm{rad}] \\text{,\\quad ha} \\quad x \\in (" .. ({0, 'a', 'b'})[i] .. ";" .. ({'a', 'b', 'c'})[i] .. ")")

    if i ~= 3 then
      tex.sprint "\\text,\\\\"
    else
      tex.sprint "\\text."
    end
  end
  tex.sprint [[\end{alignat}]]
\end{luacode*}
% \end{noindent}
\egroup

A hajlítónyomatéki függvény:
\bgroup
\sifigures{6}
% \begin{noindent}
\begin{luacode*}
  tex.sprint [[\begin{alignat}{9}]]
  for i=1,3 do
    tex.sprint("M_{h," .. i .. "}(x)&=\\,")

    local MhMat = V.MMat

    for k=2,0,-1 do
      local value = MhMat[i][k]

      if math.abs(value) >= 1e-16 then
        if value > 0 and k ~= 3 then
          tex.sprint "&+&"
        else
          tex.sprint "&-&"
        end
        H.printSIDirect {value=math.abs(value), unit="", dec=6}

        tex.sprint "&\\,"
        if k == 0 then
          tex.sprint ""
        elseif  k == 1 then 
          tex.sprint "x"
        elseif k == 2 then
          tex.sprint "x^2"
        end
        tex.sprint("\\,")
      else
        tex.sprint("&&&")
      end
    end

    tex.sprint("[&\\mathrm{Nm}] \\text{,\\quad ha} \\quad x \\in (" .. ({0, 'a', 'b'})[i] .. ";" .. ({'a', 'b', 'c'})[i] .. ")")

    if i ~= 3 then
      tex.sprint "\\text,\\\\"
    else
      tex.sprint "\\text."
    end
  end
  tex.sprint [[\end{alignat}]]
\end{luacode*}
% \end{noindent}
\egroup

