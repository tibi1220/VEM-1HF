% Link between TeX and lua files
\usepackage{luacode}

\begin{luacode}
  local getVariables = require 'lua.calc'
  V = getVariables {
      name = 'Sándor Tibor',
      neptun = 'C7XUDE',
      code = { 1, 1, 3, 2, 1 }, -- Last number should stay 1!
    }
  P = V.parametric

  local getHelper = require 'lua.tex-helper'
  H = getHelper {
      variables = V
    }
\end{luacode}

% For optional param usage
\usepackage{xargs}
% table access, no pretty print
\newcommand{\lv}[1]{\directlua{H.printVar [[#1]]}}
\newcommand{\lvec}[2]{\directlua{H.printVec { name="#1", index=#2 }}}
% table access, pretty print with <siunitx>
\newcommandx{\silv}[3][3=]{\directlua{H.printSIVar { name="#1", unit="#2", dec="#3" }}}
% raw access, no pretty print
\newcommand{\dv}[1]{\directlua{H.printDirect(#1)}}

% Optional commands for printing
\newcommandx{\F}[3][1=,3=]{\directlua{H.optionallyPrintF([[#1]], #2, [[#3]])}}
\newcommandx{\M}[3][1=,3=]{\directlua{H.optionallyPrintM([[#1]], #2, [[#3]])}}
\newcommandx{\p}[3][1=,3=]{\directlua{H.optionallyPrintP([[#1]], #2, [[#3]])}}
\newcommand{\E}[1]{\directlua{H.printParam("E", #1)}}
\newcommand{\I}[1]{\directlua{H.printParam("I", #1)}}
