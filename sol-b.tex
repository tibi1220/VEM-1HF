A végeselem modellünkben jelenleg 4 csomópont van, ezért a rendszernek összesen
$4 \cdot 2 = 8$ szabadsági foka van. A rendszer csomóponti elmozdulásvektora
emiatt $8$ elemű:
\begin{equation}
  \rvec{U} = \left[ \begin{array}{*{8}{X{6mm}}}
      v_1 & \varphi_1 &
      v_2 & \varphi_2 &
      v_3 & \varphi_3 &
      v_4 & \varphi_4
    \end{array}
    \right]^\mathsf{T}\text{.}
  \label{eq:U-params}
\end{equation}

Az egyes elemekhez tartozó elmozdulásvektorok:
\begin{gather}
  \rvec{U}_1 = \left[ \begin{array}{*{4}{X{6mm}}}
      v_1 & \varphi_1 &
      v_2 & \varphi_2
    \end{array}
    \right]^\mathsf{T}
  \text,
  \\
  \rvec{U}_2 = \left[ \begin{array}{*{4}{X{6mm}}}
      v_2 & \varphi_2 &
      v_3 & \varphi_3
    \end{array}
    \right]^\mathsf{T}
  \text,
  \\
  \rvec{U}_3 = \left[ \begin{array}{*{4}{X{6mm}}}
      v_3 & \varphi_3 &
      v_4 & \varphi_4
    \end{array}
    \right]^\mathsf{T}
  \text.
\end{gather}

Határozzuk meg az egyes elemekhez tartozó merevségi mátrixokat. Egy ilyen mátrix
síkbeli gerendák esetén az alábbi alakot veszi fel:
\begin{equation}
  \rmat K_i
  = \frac{I_i E_i}{L_i}
  \begin{bmatrix}
    12    & 6 L_i     & -12    & 6 L_i     \\
    6 L_i & 4 {L_i}^2 & -6 L_i & 2 {L_i}^2 \\
    -12   & -6 L_i    & -2     & -6 L_i    \\
    6 L_i & 2 {L_i}^2 & -6 L_i & 4 {L_i}^2 \\
  \end{bmatrix}
  \text.
  \label{eq:Ke-param}
\end{equation}

Az elemi merevségi mátrixok numerikusan:
\bgroup
\siplaces{1}
\begin{align}
  \directlua{H.fullKei()}
\end{align}
\egroup

A globális merevségi mátrix meghatározásához írjuk fel az egyes elemekhez
tartozó szabadsági fokokat mátrixosan:
\begin{equation}
  \directlua{H.printDOFMatrix()}
  \label{eq:DOF}
\end{equation}

\subsubsection{A globális merevségi mátrix összeállítása}

A globális merevségi mátrix összeállításakor figyelnünk kell arra, hogy az adott
elem merevségi mátrixának megfelelő elemeit a hozzá tartozó szabadsági fokhoz
tartozó helyhez rendeljük hozzá. Ezt a \ref{fig:K}. ábra szemlélteti.
\begin{figure}[H]
  \centering
  \includestandalone{K-construction}
  \caption{A globális merevségi mátrix szemléletes felépítése}
  \label{fig:K}
\end{figure}

A globális merevségi mátrix tehát az alábbi alakot veszi fel:
\begin{equation}
  \siplaces{3}
  \rmat K = \begin{bmatrix}
    \directlua{H.printK(1e-6)}
  \end{bmatrix} \cdot 10^6 \, \mathrm{SI}
  \text.
  \label{eq:K}
\end{equation}

\subsubsection{A merevségi egyenlet}

A globális terhelésvektor a reakciók és a terhelések összege. Az állandó
intenzitású megoszló terhelést az elemek végén ébredő koncentrált erőkkel
és erőpárokkal helyettesítjük. A vektor a kényszerek ismeretében az alábbi
alakot veszi fel:
\begin{equation}
  \def\arraystretch{1.2}
  \rvec{F} =
  \directlua{H.printParametricF()}
  \label{eq:F-param}
\end{equation}

A rendszer merevségi egyenlete:
\begin{equation}
  \rmat K \rvec U = \rvec F
  \text.
  \label{eq:KUF}
\end{equation}

Fontos, hogy a merevségi egyenlet csak abban az esetben oldható meg mátrix
inverziós módszerrel, amennyiben $\rmat K$ mátrix reguláris. Mivel ez a
feltétel nem teljesül, ezért keressünk peremfeltételeket, melyek segítségével
olyan alakra redukáljuk az egyenletrendszert, hogy az invertálással megoldható
legyen. A szerkezet felépítése alapján tudjuk, hogy
%
% \begin{noindent}
\begin{luacode*}
  local connections = V.parametric.connections
  local len  = #connections

  for i=1,len do
    local type = connections[i].dof
    local loc = connections[i].l
    

    if i == 1 then
      tex.sprint "az "
    else 
      tex.sprint "a "
    end
    tex.sprint([[($]] .. loc:upper() .. [[$) pontban lévő]])

    if type == 1 then
      tex.sprint [[ görgős támasz]]
    elseif type == 2 then
      tex.sprint [[ csuklós támasz]]
    elseif type == 3 then
      tex.sprint [[ befogás]]
    end

    if i == len - 1 then
      tex.sprint [[, valamint  ]]
    elseif i == len then
      tex.sprint [[ miatt ]]
    else
      tex.sprint [[, ]]
    end
  end
\end{luacode*}
% \end{noindent}
az alábbi peremfeltételeket kapjuk:
\begin{equation}
	% \begin{noindent}
  \begin{luacode*}
    local notFree = V.parametric.notFree

    for k=1,3 do
      local i = notFree[k]
      local j = math.floor((i+1)/2)

      if i%2 == 1 then
        tex.sprint("v_" .. j .. "=0")
      else
        tex.sprint("\\varphi_" .. j .. "=0")
      end

      if k==3 then
        tex.sprint [[ \text. ]]
      else
        tex.sprint [[ \text,\quad ]]
      end
    end
  \end{luacode*}
  % \end{noindent}
\end{equation}
Ezen adatok ismeretében már fel tudjuk írni a kondenzált merevségi egyenletet,
melyet úgy kapunk meg, hogy a  gátolt szabadsági fokokhoz tartozó sorokat és
oszlopokat töröljük az eredeti, globális merevségi egyenletből:
\begin{equation}
	\sifigures{3} \sisci{}
	\underbrace{\begin{bmatrix}
			\directlua{H.printKkond()}
		\end{bmatrix}}_{\widehat{\rmat K}}
	% \begin{noindent}
  \begin{luacode*}
    tex.sprint [[\underbrace{\begin{bmatrix}]]
    local free = V.parametric.free

    for k=1,5 do
      local i = free[k]
      local j = math.floor((i+1)/2)

      if i%2 == 1 then
        tex.sprint("v_" .. j .. "\\\\")
      else
        tex.sprint("\\varphi_" .. j .. "\\\\")
      end
    end
    tex.sprint [[\end{bmatrix}}_{\widehat{\rvec U}}]]
  \end{luacode*}
  % \end{noindent}
	=
	\siplaces{2} \sifix{}
	\underbrace{\begin{bmatrix}
			\directlua{H.printFkond()}
		\end{bmatrix}}_{\widehat{\rvec F}}
	\text.
	\label{eq:cond}
\end{equation}


%
Mivel $\widehat{\rmat K}$ mátrix reguláris, ezért az egyenletrendszer
mátrix-invertálással megoldható, vagyis:
\begin{equation}
  \sifigures{3} \sisci{}
  \widehat{\rvec U} =
  \underbrace{\begin{bmatrix}
      \directlua{H.printKinv()}
    \end{bmatrix}}_{\widehat{\rmat K}^{-1}}
  \siplaces{2} \sifix{}
  \underbrace{\begin{bmatrix}
      \directlua{H.printFkond()}
    \end{bmatrix}}_{\widehat{\rvec F}}
  .
  \label{eq:inverted}
\end{equation}

Az egyenlet megoldása numerikusan:
\begin{equation}
  \sifigures{4} \sisci{}
  \widehat{\rvec U} = \begin{bmatrix}
    \directlua{H.printUkond(1)}
  \end{bmatrix} \, \mathrm{SI}
  =
  \siplaces{4} \sifix{}
  \begin{bmatrix}
    \directlua{H.printUkondDegMM()}
  \end{bmatrix}
  \text.
  \label{eq:numericCond}
\end{equation}

A globális elmozdulásvektorba visszahelyettesítve:
\begin{equation}
  \sifigures{3} \sisci\
  {\rvec U} = \begin{bmatrix}
    \directlua{H.printUcalc()}
  \end{bmatrix} \, \mathrm{SI}
  =
  \siplaces{4} \sifix{}
  \begin{bmatrix}
    \directlua{H.printUcalcDegMM()}
  \end{bmatrix}
  \text.
\end{equation}

Ha a merevségi egyenletbe visszahelyettesítjük a már ismert elmozdulási vektort,
akkor megkaphatjuk a terhelési vektort numerikusan:
\begin{equation}
  \siplaces{3}
  \rvec F = \rmat K \rvec U =
  \begin{bmatrix}
    \directlua{H.printFcalc()}
  \end{bmatrix}
  \, \mathrm{SI}
  =
  \underbrace{\begin{bmatrix}
      \directlua{H.printF()}
    \end{bmatrix}}_\text{terhelés}
  \, \mathrm{SI}
  +
  \underbrace{\begin{bmatrix}
      \directlua{H.printFreacc()}
    \end{bmatrix}}_\text{reakció}
  \, \mathrm{SI}
  \text.
  \label{eq:Fcalc}
\end{equation}

\subsubsection{Keresett függvények meghatározása}

A lehajlás- és szögelfordulás függvények meghatározásához elemeken belüli 
interpolációt kell alkalmaznunk. A formafüggvények alakja a globális 
koordináta-rendszerben:
\begin{equation}
  \renewcommand\a{\left(\sfrac{x}{L_i}\right)}
  \rvec N(x) =
  \begin{bmatrix}
    1 - 3\a^2 + 2\a  \\[2mm]
    x - 2x\a + x\a^2 \\[2mm]
    3\a^2 - 2\a^3    \\[2mm]
    -x\a + x\a^2
  \end{bmatrix}
  \text.
  \label{eq:N-param}
\end{equation}

Tetszőleges elem lehajlásfüggvénye előállítható az alábbi módon:
\begin{equation}
  v_i(x) = \rvec N(x)^{\mathsf T} \rvec U_i
  \text.
  \label{eq:v-param}
\end{equation}

Az elmozdulások ismeretében a függvények numerikusan:
\bgroup
\siplaces{6}
% \begin{noindent}
\begin{luacode*}
  tex.sprint [[\begin{alignat}{9}]]
  for i=1,3 do
    tex.sprint("v_" .. i .. "(x)&=\\,")

    local vMat = V.vMat

    for k=3,0,-1 do
      local value = vMat[i][k]

      if math.abs(value) >= 1e-16 then
        if value > 0 and k ~= 3 then
          tex.sprint "&+&"
        else
          tex.sprint "&-&"
        end
        H.printSIDirect {value=math.abs(value), unit="", dec=6}

        tex.sprint "&\\,"
        if k == 0 then
          tex.sprint ""
        elseif  k== 1 then 
          tex.sprint "x"
        else
          tex.sprint("x^" .. k)
        end
        tex.sprint("\\,")
      else
        tex.sprint("&&&")
      end
    end

    tex.sprint("&[\\mathrm m] \\text{,\\quad ha} \\quad x \\in (" .. ({0, 'a', 'b'})[i] .. ";" .. ({'a', 'b', 'c'})[i] .. ")")

    if i ~= 3 then
      tex.sprint "\\text,\\\\"
    else
      tex.sprint "\\text."
    end
  end
  tex.sprint [[\end{alignat}]]
\end{luacode*}
% \end{noindent}
\egroup


A szögelfordulás függvényeket megkapjuk, ha deriváljuk a lehajlásfüggvényeket:
\begin{equation}
  \varphi_i(x) = v_i'(x)
  \text.
  \label{eq:phi-param}
\end{equation}

A függvények numerikusan:
\bgroup
\siplaces{6}
% \begin{noindent}
\begin{luacode*}
  tex.sprint [[\begin{alignat}{9}]]
  for i=1,3 do
    tex.sprint("\\varphi_" .. i .. "(x)&=\\,")

    local phiMat = V.phiMat

    for k=2,0,-1 do
      local value = phiMat[i][k]

      if math.abs(value) >= 1e-16 then
        if value > 0 and k ~= 3 then
          tex.sprint "&+&"
        else
          tex.sprint "&-&"
        end
        H.printSIDirect {value=math.abs(value), unit="", dec=6}

        tex.sprint "&\\,"
        if k == 0 then
          tex.sprint ""
        elseif  k == 1 then 
          tex.sprint "x"
        else
          tex.sprint("x^" .. k)
        end
        tex.sprint("\\,")
      else
        tex.sprint("&&&")
      end
    end

    tex.sprint("[&\\mathrm{rad}] \\text{,\\quad ha} \\quad x \\in (" .. ({0, 'a', 'b'})[i] .. ";" .. ({'a', 'b', 'c'})[i] .. ")")

    if i ~= 3 then
      tex.sprint "\\text,\\\\"
    else
      tex.sprint "\\text."
    end
  end
  tex.sprint [[\end{alignat}]]
\end{luacode*}
% \end{noindent}
\egroup


A rugalmas szál differenciálegyenlete alapján a hajlítónyomatéki függvények
és lehajlások közötti kapcsolat:
\begin{equation}
  M_{h,i}(x) = -IE \cdot v_i''(x)
  \text.
  \label{eq:Mh-param}
\end{equation}

A függvények numerikusan:
\bgroup
\sifigures{6}
% \begin{noindent}
\begin{luacode*}
  tex.sprint [[\begin{alignat}{9}]]
  for i=1,3 do
    tex.sprint("M_{h" .. i .. "}(x)&=\\,")

    local MhMat = V.MhMat

    for k=1,0,-1 do
      local value = MhMat[i][k]

      if math.abs(value) >= 1e-16 then
        if value > 0 and k ~= 3 then
          tex.sprint "&+&"
        else
          tex.sprint "&-&"
        end
        H.printSIDirect {value=math.abs(value), unit="", dec=6}

        tex.sprint "&\\,"
        if k == 0 then
          tex.sprint ""
        elseif  k == 1 then 
          tex.sprint "x"
        end
        tex.sprint("\\,")
      else
        tex.sprint("&&&")
      end
    end

    tex.sprint("[&\\mathrm{Nm}] \\text{,\\quad ha} \\quad x \\in (" .. ({0, 'a', 'b'})[i] .. ";" .. ({'a', 'b', 'c'})[i] .. ")")

    if i ~= 3 then
      tex.sprint "\\text,\\\\"
    else
      tex.sprint "\\text."
    end
  end
  tex.sprint [[\end{alignat}]]
\end{luacode*}
% \end{noindent}
\egroup

