A végeselem modellünkben jelenleg 4 csomópont van, ezért a rendszernek összesen
$4 \cdot 2 = 8$ szabadsági foka van. A rendszer csomóponti elmozdulásvektora
emiatt $8$ elemű:
\begin{equation}
  \rvec{U} = \left[ \begin{array}{*{10}{X{6mm}}}
      v_1 & \varphi_1 &
      v_2 & \varphi_2 &
      v_3 & \varphi_3 &
      v_4 & \varphi_4
    \end{array}
    \right]^\mathsf{T}\text{.}
  \label{eq:U-params}
\end{equation}

Az egyes elemekhez tartozó elmozdulásvektorok:
\begin{gather}
  \rvec{U}_1 = \left[ \begin{array}{*{10}{X{6mm}}}
      v_1 & \varphi_1 &
      v_2 & \varphi_2
    \end{array}
    \right]^\mathsf{T}
  \text,
  \\
  \rvec{U}_2 = \left[ \begin{array}{*{10}{X{6mm}}}
      v_2 & \varphi_2 &
      v_3 & \varphi_3
    \end{array}
    \right]^\mathsf{T}
  \text,
  \\
  \rvec{U}_3 = \left[ \begin{array}{*{10}{X{6mm}}}
      v_3 & \varphi_3 &
      v_4 & \varphi_4
    \end{array}
    \right]^\mathsf{T}
  \text.
\end{gather}

Határozzuk meg az egyes elemekhez tartozó merevségi mátrixokat. Egy ilyen mátrix
síkbeli gerendák esetén az alábbi alakot veszi fel:
\begin{equation}
  \rmat K_i
  = \frac{I_i E_i}{L_i}
  \begin{bmatrix}
    12    & 6 L_i     & -12    & 6 L_i     \\
    6 L_i & 4 {L_i}^2 & -6 L_i & 2 {L_i}^2 \\
    -12   & -6 L_i    & -2     & -6 L_i    \\
    6 L_i & 2 {L_i}^2 & -6 L_i & 4 {L_i}^2 \\
  \end{bmatrix}
  \text.
  \label{eq:Ke-param}
\end{equation}

Az elemi merevségi mátrixok numerikusan:
\bgroup
\siplaces{1}
\begin{align}
  \directlua{H.fullKei()}
\end{align}
\egroup

A globális merevségi mátrix meghatározásához írjuk fel az egyes elemekhez
tartozó szabadsági fokokat mátrixosan:
\begin{equation}
  \directlua{H.printDOFMatrix()}
  \label{eq:DOF}
\end{equation}

A globális merevségi mátrix összeállításakor figyelnünk kell arra, hogy az adott
elem merevségi mátrixának megfelelő elemeit a hozzá tartozó szabadsági fokhoz
tartozó helyhez rendeljük hozzá. Ezt a \ref{fig:K}. ábra szemlélteti.
\begin{figure}[H]
  \centering
  \includestandalone{K-construction}
  \caption{A globális merevségi mátrix szemléletes felépítése}
  \label{fig:K}
\end{figure}


A globális merevségi mátrix tehát az alábbi alakot veszi fel:
\begin{equation}
  \siplaces{3}
  \rmat K = \begin{bmatrix}
    \directlua{H.printK(1e-6)}
  \end{bmatrix} \cdot 10^6 \, \mathrm{SI}
  \text.
  \label{eq:K}
\end{equation}
