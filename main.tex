\documentclass[a4paper, 12pt]{scrartcl}

% Use Hungarian Locale
\usepackage[magyar]{babel}

% Inputs
% Link between TeX and lua files
\usepackage{luacode}

\begin{luacode}
  local getVariables = require 'lua.calc'
  V = getVariables {
      name = 'Sándor Tibor',
      neptun = 'C7XUDE',
      code = { 1, 1, 3, 2 },
    }

  local getHelper = require 'lua.tex-helper'
  H = getHelper {
      variables = V
    }
\end{luacode}

% For optional param usage
\usepackage{xargs}
\newcommand{\lv}[1]{\directlua{H.printVar [[#1]]}}
\newcommand{\lvec}[2]{\directlua{H.printVec { name="#1", index=#2 } }}

% Figure and other imports
\usepackage{pdfpages}
\usepackage{standalone}

% Actual page layout
\usepackage[
  margin=20mm,
  footskip=12mm,
  headheight=20mm,
  % showframe
]{geometry}
\usepackage{fancyhdr}
\usepackage{lastpage}
\usepackage{hyperref}
\pagestyle{fancy}

\renewcommand\footrulewidth{1mm}
\renewcommand\headrulewidth{1mm}

\setlength\parindent{0em}
\setlength\parskip{.67em}

\fancyfoot[C]{\thepage\ / \pageref{LastPage}}
\fancyhead[L]{BMEGEMMBXVE, 1. Házi Feladat}
\fancyhead[R]{%
  \includegraphics[height = 12px]{static/signature.pdf}
  \lv{name},
  \lv{neptun}
}

\renewcommand{\thesubsection}{\thesection.\alph{subsection}}



% Math related stuff goes here
\usepackage{amsmath}
\usepackage{amssymb}
\usepackage{fontspec}
\usepackage{unicode-math}

% Set font to my liking
\setmainfont{TeX Gyre Termes}
\setsansfont[Scale=MatchUppercase]{TeX Gyre Heros}
\setmathfont{Asana Math}

% Variable printing according to Hungarian standards
\usepackage{icomma}
\usepackage{siunitx}
\sisetup{
  per-mode = symbol,
  locale=DE
}

% Math custom commands
\newcommand\iu{\mathbf{j}}
\newcommand{\rvec}[1]{\mathbfit{#1}}
\newcommand{\uvec}[1]{\widehat{\mathbfit{#1}}}
\newcommand{\rmat}[1]{\mathbf{#1}}

% Operator like commands
\DeclareMathOperator\atann{atan2}

% Switching between SI modes
\newcommand{\sifigures}[1]{\sisetup{exponent-mode = fixed, round-mode=figures, round-precision=#1}}
\newcommand{\siplaces}[1]{\sisetup{exponent-mode = fixed, round-mode=places, round-precision=#1}}
\newcommand{\sisci}{\sisetup{exponent-mode = scientific}}

\usepackage{tikz}
\usepackage{pgfplots}
\pgfplotsset{compat=1.18, width=16cm, height=7.667cm}
\usepgfplotslibrary{fillbetween}
\usetikzlibrary{
  calc,
  angles,
  quotes,
  backgrounds,
  patterns,
  arrows,
  arrows.meta,
  positioning,
  intersections,
  shapes.geometric,
}

\tikzset{
  dot/.style = {
      circle,
      fill=red!80!gray,
      minimum size=#1,
      draw=black,
      inner sep=0pt, outer sep=0pt,
      very thick,
    },
  dot/.default = 7pt,
  gdot/.style = {
      dot,
      fill=white
    },
  dim/.style = {
      latex-latex,
      draw=teal,
      ultra thick
    },
  joint/.style = {
      circle,
      draw=black,
      ultra thick,
      fill=cyan!20,
      minimum size=4mm,
    },
  square/.style = {
      regular polygon,
      regular polygon sides=4
    },
  rod/.style = {
      rectangle,
      draw=black,
      minimum height=6mm,
      minimum width=6mm,
      fill=yellow!10,
      ultra thick,
      midway,
      outer sep=0,
    },
}




\input{config/other}

\begin{document}

\AddToShipoutPictureFG*{
  \setmainfont{Latin Modern Roman}
  \setmathfont{Latin Modern Math}
  \put(12cm,27.33cm){
    \makebox(7.5cm,1cm){
      \hfill \lv{name}
    }
  }
  \put(12cm,26.67cm){
    \makebox(7.5cm,1cm){
      \hfill \lv{neptun}
    }
  }
  \put(15.45cm,26cm){
    \makebox(7.5cm,1cm){
      \hfill \includegraphics[height=5mm]{./static/signature.pdf} \hfill
    }
  }

  \put(9.5cm,25cm){
    \makebox(15mm,0){\lvec{code}{1}}
    \makebox(8.75mm,0){\lvec{code}{2}}
    \makebox(8.95mm,0){\lvec{code}{3}}
    \makebox(8.75mm,0){\lvec{code}{4}}
  }
}
\includepdf[
  pages=-,
  scale=.95,
  pagecommand=\thispagestyle{fancy}
]{static/titlepage.pdf}
\setmainfont{TeX Gyre Termes}
\setmathfont{Asana Math}


\section{Méretarányos ábra} % (fold)
\label{sec:Méretarányos ábra}

A feladatkódom (\texttt{\lvec{code}{1}\lvec{code}{2}\lvec{code}{3}\lvec{code}{4}})
alapján a szerkezetet jellemző adatok:
\begin{multicols}{3}
  \begin{itemize}
    \item $a = \silv{a}{m}$,
    \item $b = \silv{b}{m}$,
    \item $c = \silv{c}{m}$,

    \item $p_1 = p_t = \silv{p_t}{N/m}$.
    \item $F_1 = F_t = \silv{F_t}{N}$.
    \item $M_1 = M_t = \silv{M_t}{Nm}[0]$,

    \item $\siplaces{0}\sisci{}E = \silv{E}{Pa}$,
    \item $d = \silv{d}{m}$.
  \end{itemize}
\end{multicols}

A megadott adatok alapján a szerkezetről készített méretarányos vázlat az
\ref{fig:construction}. ábrán látható. A lapon mért $\SI{1}{cm}$ távolság
a valóságban $\SI{10}{cm}$-nek felel meg.

\begin{figure}[H]
  \centering
  \includestandalone{construction}
  \caption{Méretarányos ábra a tartóról}
  \label{fig:construction}
\end{figure}

% section Méretarányos ábra (end)



\section{Lehajlás-, és hajlítónyomatéki igénybevétel függvények} % (fold)
\label{sec:Lehajlás- és hajlítónyomatéki igénybevétel függvények}

\begin{figure}[H]
  \centering
  \includestandalone{numbering}
  \vspace{-2mm}
  \caption{A csomópontok és elemek számozása}
  \label{fig:numbering}
\end{figure}

\vspace{-2mm}
A keresett függvényeket először szilárdságtani, majd pedig végeselemes
módszer segítségével fogjuk meghatározni. Számítsuk ki először azokat a
paramétereket, melyeket mindkét megoldás fel fogunk használni.

\bgroup
\begin{multicols}{2}
  A vékonyabb gerenda paraméterei:
  \begin{itemize}
    \item $d_1 = d = \silv{d_1}{m}$
          \sisci{} \siplaces{0}
    \item $E_1 = 4E = \silv{E_1}{Pa}$
          \siplaces{4}
    \item $A_1 = \dfrac{{d_1}^2 \pi}{4} = \silv{A_1}{m^2}$
          \siplaces{4}
    \item $I_1 = \dfrac{{d_1}^4 \pi}{64} = \silv{I_1}{m^4}$
  \end{itemize}
  A vastagabb gerenda paraméterei:
  \begin{itemize}
    \item $d_2 = 2d = \silv{d_2}{m}$
          \sisci{} \siplaces{0}
    \item $E_2 = E = \silv{E_2}{Pa}$
          \siplaces{4}
    \item $A_2 = \dfrac{{d_2}^2 \pi}{4} = \silv{A_2}{m^2}$
          \siplaces{4}
    \item $I_2 = \dfrac{{d_2}^4 \pi}{64} = \silv{I_2}{m^4}$
  \end{itemize}
\end{multicols}
\egroup

Az egyes elemekhez és csomópontokhoz hozzárendelt sorszámok a
\ref{fig:numbering}. ábra tartalmazza.

A csomópontok koordinátáit az \ref{table:U}. táblázat tartalmazza.
\begin{table}[H]
  \def\arraystretch{1.1}
  \centering
  \caption{A csomópontok koordinátái}
  \begin{tabular}{| c || X{1.5cm} | X{1.5cm} |}
    \hline
    Csp. & x \,[\text{m}] & y \,[\text{m}]
    \\ \hline \hline
    1    & \SI{0}{m}      & \SI{0}{m}
    \\ \hline
    2    & \silv{a}{m}    & \SI{0}{m}
    \\ \hline
    3    & \silv{b}{m}    & \SI{0}{m}
    \\ \hline
    4    & \silv{c}{m}    & \SI{0}{m}
    \\ \hline
  \end{tabular}
  \label{table:U}
\end{table}

Az egyes elemekhez tartozó csomópontokat, valamint további paramétereiket
a \ref{table:lok}. táblázat foglalja össze.
\begin{table}[H]
  \def\arraystretch{1.1}
  \centering
  \caption{Elem -- csomópont összerendelések}
  \begin{tabular}{| c || c | c || *{5}{>{$}c<{$}|}}
    \hline
    Elem & 1. csp & 2. csp & d                          & L         & A                          & E                          & I                          \\ \hline \hline
    1    & 1      & 2      & d_\dv{P.beams[1]['value']} & L_1 = a-0 & A_\dv{P.beams[1]['value']} & E_\dv{P.beams[1]['value']} & I_\dv{P.beams[1]['value']} \\ \hline
    2    & 2      & 3      & d_\dv{P.beams[2]['value']} & L_2 = b-a & A_\dv{P.beams[2]['value']} & E_\dv{P.beams[2]['value']} & I_\dv{P.beams[2]['value']} \\ \hline
    3    & 3      & 4      & d_\dv{P.beams[3]['value']} & L_3 = c-b & A_\dv{P.beams[3]['value']} & E_\dv{P.beams[3]['value']} & I_\dv{P.beams[3]['value']} \\ \hline
  \end{tabular}
  \label{table:lok}
\end{table}




\subsection{Megoldás differenciálegyenlettel} % (fold)
\label{ssec:Megoldás differenciálegyenlettel}

Írjuk fel az $y$ és $z$ irányú egyensúlyi egyenleteket:




% subsection Megoldás differenciálegyenlettel (end)


\subsection{Megoldás végeselem módszerrel} % (fold)
\label{ssec:Megoldás végeselem módszerrel}

A végeselem modellünkben jelenleg 4 csomópont van, ezért a rendszernek összesen
$4 \cdot 2 = 8$ szabadsági foka van. A rendszer csomóponti elmozdulásvektora
emiatt $8$ elemű:
\begin{equation}
  \rvec{U} = \left[ \begin{array}{*{10}{X{6mm}}}
      v_1 & \varphi_1 &
      v_2 & \varphi_2 &
      v_3 & \varphi_3 &
      v_4 & \varphi_4
    \end{array}
    \right]^\mathsf{T}\text{.}
  \label{eq:U-params}
\end{equation}

Az egyes elemekhez tartozó elmozdulásvektorok:
\begin{gather}
  \rvec{U}_1 = \left[ \begin{array}{*{10}{X{6mm}}}
      v_1 & \varphi_1 &
      v_2 & \varphi_2
    \end{array}
    \right]^\mathsf{T}
  \text,
  \\
  \rvec{U}_2 = \left[ \begin{array}{*{10}{X{6mm}}}
      v_2 & \varphi_2 &
      v_3 & \varphi_3
    \end{array}
    \right]^\mathsf{T}
  \text,
  \\
  \rvec{U}_3 = \left[ \begin{array}{*{10}{X{6mm}}}
      v_3 & \varphi_3 &
      v_4 & \varphi_4
    \end{array}
    \right]^\mathsf{T}
  \text.
\end{gather}

Határozzuk meg az egyes elemekhez tartozó merevségi mátrixokat. Egy ilyen mátrix
síkbeli gerendák esetén az alábbi alakot veszi fel:
\begin{equation}
  \rmat K_i
  = \frac{I_i E_i}{L_i}
  \begin{bmatrix}
    12    & 6 L_i     & -12    & 6 L_i     \\
    6 L_i & 4 {L_i}^2 & -6 L_i & 2 {L_i}^2 \\
    -12   & -6 L_i    & -2     & -6 L_i    \\
    6 L_i & 2 {L_i}^2 & -6 L_i & 4 {L_i}^2 \\
  \end{bmatrix}
  \text.
  \label{eq:Ke-param}
\end{equation}

Az elemi merevségi mátrixok numerikusan:
\bgroup
\siplaces{1}
\begin{align}
  \directlua{H.fullKei()}
\end{align}
\egroup

A globális merevségi mátrix meghatározásához írjuk fel az egyes elemekhez
tartozó szabadsági fokokat mátrixosan:
\begin{equation}
  \directlua{H.printDOFMatrix()}
  \label{eq:DOF}
\end{equation}

A globális merevségi mátrix összeállításakor figyelnünk kell arra, hogy az adott
elem merevségi mátrixának megfelelő elemeit a hozzá tartozó szabadsági fokhoz
tartozó helyhez rendeljük hozzá. Ezt a \ref{fig:K}. ábra szemlélteti.
\begin{figure}[H]
  \centering
  \includestandalone{K-construction}
  \caption{A globális merevségi mátrix szemléletes felépítése}
  \label{fig:K}
\end{figure}


A globális merevségi mátrix tehát az alábbi alakot veszi fel:
\begin{equation}
  \siplaces{3}
  \rmat K = \begin{bmatrix}
    \directlua{H.printK(1e-6)}
  \end{bmatrix} \cdot 10^6 \, \mathrm{SI}
  \text.
  \label{eq:K}
\end{equation}


% subsection Megoldás végeselem módszerrel (end)


\clearpage
\subsection{Megoldásfüggvények ábrázolása} % (fold)
\label{sub:Megoldásfüggvények ábrázolása}

\begin{figure}[H]
  \centering
  \includestandalone{plot-v}
  \caption{$v(x)$}
  \label{fig:plot-v}
\end{figure}

\begin{figure}[H]
  \centering
  \includestandalone{plot-phi}
  \caption{$\varphi(x)$}
  \label{fig:plot-phi}
\end{figure}

\begin{figure}[H]
  \centering
  \includestandalone{plot-Mh}
  \caption{$M_h(x)$}
  \label{fig:plot-Mh}
\end{figure}


% subsection Megoldásfüggvények ábrázolás (end)



% section Lehajlás-, és hajlítónyomatéki igénybevétel függvények (end)

\end{document}
